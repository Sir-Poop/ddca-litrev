\section{ Demand-driven deployment models}
To predict the market demand for nuclear fuel cycle simulators, various algorithms
can be adopted. The algorithms can be divided into three major
branches, Non-optimizing, Deterministic Optimizing, and Stochastic
Optimizing. Each groups vary in complexity, which leads to a difference
in computing time and accuracy. The groups and their examples are 
explained in detail below. 

\subsection{Non-Optimizing(NO)}

Non optimizing algorithms predict future deployment schedules
only based on the past datasets. These algorithms do not attempt
to minimize the difference of the produced quantity to the demand,
thus the name non-optimizing. The simple nature of the algorithm
allows fast execution time, but a drawback of low precision. The
two main algorithms explored are \gls{ARMA}
and \gls{ARCH} methods. 

\subsubsection{Autoregressive Moving Average (ARMA)}

\gls{ARMA} is a combination of two methods, the Autoregressive
and the Moving Average model. The Autoregressive model 
predicts future values by fitting a linear curve fit of the latest
datasets, and the Moving Average method does so by fitting the 
errors. \cite{intro_arma}

The model is referred to as ARMA(p,q) format, where the p and q represent 
the order of the autoregressive, and the moving average part,
respectively. 
The \gls{ARMA} model can be represented by the following equation:

\begin{align} 
	X_i=c + \epsilon_t + {\mathlarger{\sum}}_{i=1}^{p}\theta_i \cdot X_{t-i}+
	 {\mathlarger{\sum}}_{j=1}^{q}\phi_j\cdot\epsilon_{t-j}\\
	X_i= \mbox{next predicted value of the time series}\\
	c= \mbox{adjustable constant}\\
	\theta_i=i^{th} \mbox{parameter}\\
	\phi_i=j^{th} \mbox{parameter of the moving average system}\\
	\epsilon = \mbox{white noise}
\end{align}


\gls{ARMA} is applied to a 'well behaved' set of time series data,
where there is little volatility. This makes \gls{ARMA} a suitable
candidate for demand prediction for constant increase of number of
reactors (caused by a constant increase of power demands).


\subsubsection{Autoregressive Conditional Heteroskedastic (ARCH)}
The \gls{ARCH} model is similar to the \gls{ARMA} model, except that
it uses previous variance terms to calculate current error terms, instead
of the value itself. This allows the model to be used in high volatile 
time series like prediction of inflation or stock prices over time \cite{bollerselv_1986}.
The \gls{ARCH} model can be represented by the following equation:

\begin{align}
	\epsilon_t^2 = \alpha_0+{\mathlarger{\sum}}_{i=1}^{q}\alpha_i\cdot\epsilon_{t-1}^2\\
	\alpha_i= i^{th} \mbox{parameter of the conditional heteroskedastic model}
\end{align}

Since the demand trend interested in current \gls{NFC} simulations 
(simple transition scenarios with constant increase of energy demand) are not
volatile, the \gls{ARCH} model is not the most suitable
model for \gls{NFC} simulations. However, if a simple transition
scenario became volatile, for example by considering economics,
\gls{ARCH} may become the more preferred model over the \gls{ARMA} model.




\subsection{Deterministic-Optimizing (DO)}
Soon to be added.





