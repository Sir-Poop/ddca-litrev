\section{ Demand-driven deployment models}
To predict the market demand inside of nuclear fuel cycle simulators, various algorithms
can be implemented. These can be divided into three major
branches, Non-optimizing, Deterministic Optimizing, and Stochastic
Optimizing. Each vary in complexity, which leads to a difference
in computing time and accuracy. The groups and examples of each are 
explained in detail below. 

\subsection{Non-Optimizing(NO)}

Non-optimizing algorithms predict future deployment schedules
only based on historical supply-demand data of the simulator. These algorithms do not attempt
to minimize the difference of the produced quantity to the demand,
thus the name non-optimizing. The simple nature of this class of algorithms
allows fast execution time, but only limited precision. The
two non-optimizing algorithms explored in this paper are \gls{ARMA}
and \gls{ARCH} methods. 

\subsubsection{Autoregressive Moving Average (ARMA)}

\gls{ARMA} is a combination of two models, the Autoregressive
and the Moving Average model. The Autoregressive model 
predicts future values with a linear curve fit of the latest
datasets, and the Moving Average method does so by fitting the 
errors. \cite{intro_arma}

The model is referred to as ARMA(p,q), where the p and q represent 
the order (number of previous time step values fitted) of the autoregressive,
and the moving average parts,
respectively. 
The \gls{ARMA} model can be represented by the following equation:

\begin{align} 
	X_i=c + \epsilon_t + {\mathlarger{\sum}}_{i=1}^{p}\theta_i \cdot X_{t-i}+
	 {\mathlarger{\sum}}_{j=1}^{q}\phi_j\cdot\epsilon_{t-j}\\
	X_i= \mbox{next predicted value of the time series}\\
	c= \mbox{adjustable constant}\\
	\theta_i=i^{th} \mbox{parameter}\\
	\phi_i=j^{th} \mbox{parameter of the moving average system}\\
	\epsilon = \mbox{white noise}
\end{align}


\gls{ARMA} is applied to a 'well behaved' set of time series data,
where there is little volatility. This makes \gls{ARMA} a suitable
candidate for demand prediction for constant increase of number of
reactors (caused by a constant increase of power demands).


\subsubsection{Autoregressive Conditional Heteroskedastic (ARCH)}
The \gls{ARCH} model is similar to the \gls{ARMA} model, except that
it uses previous variance terms to calculate current error terms, instead
of the value itself. This allows the model to be used in high volatile 
time series like prediction of inflation or stock prices over time \cite{bollerselv_1986}.
The \gls{ARCH} model can be represented by the following equation:

\begin{align}
	\epsilon_t^2 = \alpha_0+{\mathlarger{\sum}}_{i=1}^{q}\alpha_i\cdot\epsilon_{t-1}^2\\
	\alpha_i= i^{th} \mbox{parameter of the conditional heteroskedastic model}
\end{align}


In a general sense, a comprehensive fuel cycle simulator would require
predictive capabilities for volatile environments, such as economics.
In such a case, the \gls{ARCH} model may become the more preferred model
over the \gls{ARMA} model.



\subsection{Deterministic-Optimizing (DO)}
Soon to be added.





