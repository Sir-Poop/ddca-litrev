\section{Methods}

This work reviewed the international state of the art regarding dynamic 
facility deployment in \gls{NFC} simulators. It also investigated promising 
algorithmic innovations that have been successful for similar applications in 
other domains such as economics and industrial engineering.

At a high level, a meta-review of previous \gls{NFC} gap analyses 
helped to identify the existing simulators and their high 
level capabilities. The strongest comparison of transition scenario code 
capabilities is found in \cite{boucher_international_2010}, 
\cite{brown_identification_2016} and \cite{mccarthy_benchmark_2012}, in which 
international \gls{NFC} simulators to conduct specific transition scenarios 
were tested through systematic benchmarks. In 
\cite{carre_overview_2016,hoffman_expanded_2016}, the ability of individual 
simulators to conduct transition scenarios is addressed, however the 
flexibility and performance of their varying algorithms for this capability are 
not addressed.

Primary references for an array of fuel cycle simulators were consulted to 
determine the facility deployment logic present in the rest of the many 
\gls{NFC} simulators in use. Where the details of dynamic demand driven 
simulation were unknown, this review individually investigated avilable tools. 
