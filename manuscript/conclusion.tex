\section{Conclusions}
% summarize the conclusions of the paper as a list of short phrases or
% sentences. Do not repeat the results section unless special emphasis is
% needed. The conclusion is just that, not a summary. It should add a new,
% higher level of analysis and should explicitly indicate the significance of
% the work.
% What does it all mean?
% What hypotheses were proven or disproven?
% What was learned?
% Why does it make a difference?


        The review concludes that fuel cycle simulation tools approach scenario objective functions in 
        various ways.  Some wrap realizations of the simulator in an external 
        optimizer, while others employ look-ahead methods to predict malformed 
        simulation inputs.  These methods fail to realistically model the 
        process by which utilities, governments, and other stakeholders 
        actually make facility deployment decisions. Dynamic, demand-driven 
        facility deployment may be enabled by non-optimizing algorithms such as 
        \gls{ARMA} \cite{woodard_stationary_2011} and \gls{ARCH} 
        \cite{li_kernel_2016}, deterministically 
        optimizing methods such as those collected in \gls{GCAM} 
        \cite{edmonds_advanced_1994} and \gls{MARKAL} 
        \cite{fishbone_markal_1981}, or stochastic optimization 
        techniques such as Markov Switching Models \cite{ansari_predicting_2015}.
