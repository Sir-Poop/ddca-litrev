\section{Objectives}
        Nuclear fuel cycle simulation scenarios are most naturally described as 
        constrained objective functions. The objectives are often systemic 
        demands such as ``achieve 1\% growth for total electricity production 
        and reach 10\% uranium utilization''. The constraints typically take 
        the form of nuclear fuel cycle and reactor technology availability 
        (``reprocessing begins after 2025 and fast reactors first become 
        available in 2050'').

        To match the natural constrained objective form of the scenario 
        definition, \gls{NFC} simulators must bring demand responsive 
        deployment decisions into the dynamics of the simulation logic.

        In particular, a \gls{NFC} simulator should have the 
        capability to deploy supporting fuel cycle facilities to enable 
        a demand to be met. Take, for instance, the standard once through fuel 
        cycle. Reactors may clearly be deployed to meet a certain power demand. 
        However, new mines, mills, and enrichment facilities will also need to be 
        deployed to ensure that reactors have sufficient fuel to run once they 
        begin to produce power. In most simulators, the current and unrealistic 
        solution to this problem is to simply have infinite capacity support 
        facilities. Or, alternatively, detailing the deployment timeline of all 
        facilities becomes the responsibility of the user.

        The authors seek to identify the most flexible, general, and performant 
        algorithms applicable to this modeling challenge.  Accordingly, a review was 
        conducted of current \gls{NFC} simulation tools, to determine the 
        current capabilites for a demand-driven scenario formulation.  
        Additionally, promising algorithms and methods from domains outside of 
        nuclear fuel cycle simulation were investigated for applicability to 
        this problem. This paper summarizes the review method, a summary of the 
        state of the art, and an assessment of the most promising algorithms.
