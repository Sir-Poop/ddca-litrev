\section{Objectives}
        Nuclear fuel cycle simulation scenarios are most naturally described as 
        constrained objective functions. The objectives are often systemic 
        demands such as ``achieve 1\% growth for total electricity production 
        and reach 10\% uranium utilization''. The constraints typically take 
        the form of nuclear fuel cycle and reactor technology availability 
        (``reprocessing begins after 2025 and fast reactors first become 
        available in 2050'').

        Fuel cycle simulation tools approach such objective functions in 
        various ways.  Some wrap realizations of the simulator in an external 
        optimizer, while others employ look-ahead methods to predict malformed 
        simulation inputs.  These methods fail to realistically model the 
        process by which utilities, governments, and other stakeholders 
        actually make facility deployment decisions.  

        To match the natural constrained objective form of the scenario 
        definition, \gls{NFC} simulators must bring demand responsive 
        deployment decisions into the dynamics of the simulation logic.
        
        The authors seek to identify the most flexible, general, and performant 
        algorithms applicable to this purpose.  Accordingly, a review was 
        conducted of current \gls{NFC} simulation tools, to determine the 
        current capabilites for a demand-driven scenario formulation.  
        Additionally, promising algorithms and methods from domains outside of 
        nuclear fuel cycle simulation were investigated for applicability to 
        this problem. This paper summarizes the review method, a summary of the 
        state of the art, and an assessment of the most promising algorithms.

