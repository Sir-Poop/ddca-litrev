\begin{abstract}
Nuclear fuel cycle simulation scenarios are most naturally described as drivers
coupled with available technologies. These drivers are often system demands such as
``achieve 1\% growth for total electricity production and reach 10\% uranium
utilization''. The available nuclear fuel cycle and reactor technologies often
act as constraints ''reprocessing begins after 2025 and fast reactors first
become available in 2050''.

Fuel cycle simulation tools approach such objective functions in various ways.
Some wrap realizations of the simulator in an external optimizer, while others
employ look-ahead methods to predict simulation malformation.

The authors seek to bring demand and deployment decisions into the \gls{NFC}
simulator itself, thereby simulating a more realistic process by which
utilities, governments, and other stakeholders actually make facility
deployment decisions. 
To inform this process, a review was conducted of current \gls{NFC} simulation 
tools, to determine the current capabilites for a demand-driven scenario 
formulation. This paper summarizes that review.
\end{abstract}
